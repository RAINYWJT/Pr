% 请确保文件编码为utf-8,使用XeLaTex进行编译,或者通过overleaf进行编译

\documentclass[answers]{exam}  % 使用此行带有作答模块
% \documentclass{exam} % 使用此行只显示题目

\usepackage{xeCJK}
\usepackage{zhnumber}
\usepackage{graphicx}
\usepackage{hyperref}
\usepackage{amsmath}
\usepackage{amssymb}
\usepackage{mathtools}
\usepackage{booktabs}
\usepackage{enumerate}
\usepackage{enumitem} % 控制列表样式

\title{2025模式识别 \\ 作业二}
\date{2025.3.25}
\pagestyle{headandfoot}
\author{人工智能学院 221300079 王俊童}
\firstpageheadrule
\firstpageheader{南京大学}{2025模式识别}{作业二}
\runningheader{南京大学}
{2025模式识别}
{作业二}
\runningheadrule
\firstpagefooter{}{第\thepage\ 页(共\numpages 页)}{}
\runningfooter{}{第\thepage\ 页(共\numpages 页)}{}

% no box for solutions
% \unframedsolutions
\def \x \mathbf{x}


\setlength\linefillheight{.5in}

% \renewcommand{\solutiontitle}{\noindent\textbf{答:}}
\renewcommand{\solutiontitle}{\noindent\textbf{解:}\par\noindent}

\renewcommand{\thequestion}{\zhnum{question}}
\renewcommand{\questionlabel}{\thequestion .}
\renewcommand{\thepartno}{\arabic{partno}}
\renewcommand{\partlabel}{\thepartno .}

\def\dist{{\mathrm{dist}}}
\def\x{{\boldsymbol{x}}}
\def\w{{\boldsymbol{w}}}


\begin{document}
% \normalsize
\maketitle
221300079 王俊童,人工智能学院
\section{问题一}
\begin{enumerate}[label=\alph*.] 
    \item 对于这个形式化问题,根据题目的描述如下:$\gamma_{ij}$表示第j个样本(共M
    个被分到了第i类别(共K类)。而$\mu_i$代表这个类别的均值。那很显然的一个事情是,我们的
    分类标准应该是类内是要尽量小的,意思是离均值点的距离应当近。因此我们的目标是,对于每一个类别
    都要做到这一点,那对应样本应该被分进正确的类别以达到最小,问题形式化如下:
    \begin{equation*}
        \mathcal{L} =\sum_{i=1}^{K}\sum_{j=1}^{M}\gamma_{ij}\|x_j-\mu_i\|
    \end{equation*}
    那么优化函数就可以写为题目说的那个样子:
    \begin{equation}
        \label{eq:1}
        \mathop{\arg\min}_{\gamma_{ij},\mu_i} \sum_{i=1}^{K}\sum_{j=1}^{M}\gamma_{ij}\|x_j-\mu_i\|
    \end{equation}
    QED,Eqn.~\ref{eq:1}

    \item 根据题目描述,第一步应该是$\mu_i$被固定,那么优化问题就变成了对于M个样本,我们要做到优化对于每个样本来说:
    \begin{equation*}
        \mathcal{L}_j = \sum_{i=1}^{K} \gamma_{ij}\|x_j - \mu_i\|^2
    \end{equation*}
    要做到上述的j个(共M个式子)最小,意思是属于某一个类别。那么对于这个东西,相当于求一个
    \begin{equation*}
        \mathop{\arg\min}_{i} \|x_j - \mu_i\|^2
    \end{equation*}
    只要找到了最小的,其他的标识变量都是0了。所以$\gamma_{ij}$可以重写为:
    \begin{equation}
        \gamma_{ij} = 
        \begin{cases}
            1, if\ dist = min_i \|x_j - \mu_i\|^2\\
            0, otherwise.
        \end{cases}
    \end{equation}
    对于第二步骤,其实是固定了类别之后,需要重新找一个分类的最小均值。此时问题变为:
    \begin{equation*}
        \label{eq:2}
        \mathcal{L}_i = \sum_{j=1}^{M}  \gamma_{ij}\|x_j - \mu_i\|^2
    \end{equation*}
    可以求导:
    \begin{equation*}
        \frac{\partial \mathcal{L}}{\partial \mu_i} = 0
    \end{equation*}
    可以解得:
    \begin{equation}
        \label{eq:3}
        \mu_i = \frac{\sum_{j=1}^{M}\gamma_{ij}x_j}{\sum_{j=1}^{M}\gamma_{ij}}
    \end{equation}
    所以这个题目的更新规则如Eqn.~\ref{eq:2},Eqn.~\ref{eq:3}所示。
    \item 证明敛散性,相当于证明,新得到的$\mathcal{L}' - \mathcal{L}$跟0的关系:\\
    对于步骤1,样本找到了一个新的类别,意思是找到了一个更小的均值点和他的距离,那么问题可以写为:
    \begin{equation*}
        \mathcal{L}' - \mathcal{L} = \sum_{i=1}^{K}\sum_{j=1}^{M}(\gamma_{ij}' - \gamma_{ij})\|x_j-\mu_i'\|^2
    \end{equation*}
    \begin{equation*}
        \mathcal{L}' - \mathcal{L} = \sum_{i=1}^{K}\sum_{j=1}^{M}(\|x_j-\mu_i'\|^2 - \|x_j-\mu_i\|^2)\leq 0
    \end{equation*}
    显然,根据刚才的证明,他会更小,所以一定成立。\\
    对于步骤2,只针对每一个类别来说,切换新的均值点,不增加计算成本,那么考虑一个类就行了:
    \begin{equation*}
        \mathcal{L}' - \mathcal{L} \sim \sum_{j=1}^{M}(\|x_j-\mu_i'\|^2 - \|x_j-\mu_i\|^2)
    \end{equation*}
    \begin{equation*}
        \mathcal{L}' - \mathcal{L} \sim \sum_{j=1}^{M}(\mu_i - \mu_i')^\top(2x_j - \mu_i - \mu_i')
    \end{equation*}
    将Eqn.~\ref{eq:3}带入可以化简得到:
    \begin{equation*}
        \mathcal{L}' - \mathcal{L} \sim -(\mu_i - \mu_i')^\top(\mu_i - \mu_i')\leq 0
    \end{equation*}
    那么,根据证明,确实收敛。
\end{enumerate}

\section{问题二}
\begin{enumerate}[label=\alph*.] 
    \item 把题目说的形式带进去可以了:
    \begin{equation*}
        \sum_{i=1}^{n} \epsilon_i^2 = \sum_{i=1}^{n} (y_i - \x_i^\top \mathbf{\beta})
    \end{equation*}
    \begin{equation*}
        \mathop{\arg\min}_{\mathbf{\beta}} \sum_{i=1}^{n} (y_i - \x_i^\top \mathbf{\beta})
    \end{equation*}
    \item 可以写成矩阵形式的优化问题:
    \begin{equation*}
        \mathop{\arg\min}_{\mathbf{\beta}} (\mathbf{y} - \mathbf{X} \mathbf{\beta})^\top (\mathbf{y} - \mathbf{X} \mathbf{\beta})
    \end{equation*}
    矩阵形式的优化问题就是这样。
    \item 可逆的话,那么用求导等于0就好了:
    \begin{equation*}
        \frac{\partial \mathcal{L}}{\partial \beta} = 2\mathbf{X}^\top (\mathbf{X}\beta - \mathbf{y}) = 0
    \end{equation*}
    \begin{equation*}
        \beta^* = (\mathbf{X}^\top\mathbf{X})^{-1} \mathbf{X}^\top \mathbf{y}
    \end{equation*}
    \item 因为维度d比n大,第二问说X是n*d的,那么$rank(\mathbf{X})\leq n<d$可以得到,那么$rank(\mathbf{X}^\top \mathbf{X})\leq n<d$可以得到。但是,
    $\mathbf{X}^\top\mathbf{X}$这个东西是d*d的,所以一定不满秩。那就不可逆了
    \item 在正则化项加入了之后,这个东西一般不可逆的,就可以求解了。而且一般模型都没有闭式解,这种
    情况下引入正则化项既降低了复杂度,也同时使得式子可以求解,这样就有闭式解了,一般为复杂度最低的,
    感觉也算是一种归纳偏好吧。
    \item 引入正则化之后的优化目标如下:
    \begin{equation*}
        \mathop{\arg\min}_{\mathbf{\beta}} (\mathbf{y} - \mathbf{X} \mathbf{\beta})^\top (\mathbf{y} - \mathbf{X} \mathbf{\beta}) + \lambda \mathbf{\beta}^\top \mathbf{\beta}
    \end{equation*}
    求导化简:
    \begin{equation*}
        \frac{\partial \mathcal{L}}{\partial \beta} = 2\mathbf{X}^\top (\mathbf{X}\beta - \mathbf{y}) + 2\lambda \mathbf{\beta}= 0
    \end{equation*}
    \begin{equation*}
        \beta^* = (\mathbf{X}^\top\mathbf{X} + \lambda \mathbf{I})^{-1} \mathbf{X}^\top \mathbf{y}
    \end{equation*}
    \item 上面说了,如果不可逆的时候,这样就解决不了了,因为没有唯一的闭式解了。但是这个东西加入之后,那个$\lambda\mathbf{I}$可以
    让基本所有情况都可以解。这个时候就可以选择了。
    \item $\lambda=0$,就变成之前的普通线性回归,$\lambda=\infty$,这个只能解后面的$\beta^\top\beta = 0$,答案就是0.
    \item 我觉得不行,假如我们可以找到最优解$\lambda^*, \beta^*$,那么原来的运算可以写作:
    \begin{equation*}
        (\mathbf{y} - \mathbf{X} \mathbf{\beta^*})^\top (\mathbf{y} - \mathbf{X} \mathbf{\beta^*}) + \lambda^* \mathbf{\beta^*}^\top \mathbf{\beta^*}
    \end{equation*}
    但是你会发现一个非常严肃的问题,当没有正则化的时候,一定存在:
    \begin{equation*}
        (\mathbf{y} - \mathbf{X} \mathbf{\beta^*})^\top (\mathbf{y} - \mathbf{X} \mathbf{\beta^*}) < (\mathbf{y} - \mathbf{X} \mathbf{\beta^*})^\top (\mathbf{y} - \mathbf{X} \mathbf{\beta^*}) + \lambda^* \mathbf{\beta^*}^\top \mathbf{\beta^*}
    \end{equation*}
    那不是,我令为0不就行了,所以没啥用。
\end{enumerate}


\section{问题三}
\begin{enumerate}[label=\alph*.] 
    \item 1
\end{enumerate}

\end{document}